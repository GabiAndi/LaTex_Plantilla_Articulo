\chapter{Capítulo de ejemplo}

\pagestyle{empty}

\hrule
\vspace*{0.3cm}
\localtableofcontents
\vspace*{0.3cm}
\hrule

\section*{Descripción del capítulo}

Descripción

\newpage

\pagestyle{fancy}

\section{Sección}

Este capítulo es para mostrarle como es la plantilla \cite{EXAMPLE_BOOK}.

\subsection{Subsección}

Puede hacer cajas de colores:

\begin{tcolorbox}[colback=blue!5!white,colframe=blue!75!black,title=Aca iría el título]
    Y bueno aca el contenido de la caja jajajaj.
\end{tcolorbox}

\subsubsection{Subsubsección}

Tambíen pude definir tipos de cajas para reutilizarlos:

\newtcbtheorem[auto counter,number within=section]{theo}%
    {Teorema}{fonttitle=\bfseries\upshape, fontupper=\slshape,
    arc=0mm, colback=blue!5!white,colframe=blue!75!black}{teorema}

\begin{theo}{Sumatoria de números}{id_para_reconocer}
    Para todo $n$ natural:

    \begin{equation}
        \sum\limits_{i=1}^n i = \frac{n(n+1)}{2}
    \end{equation}
\end{theo}

Luego decimos que en \ref{teorema:id_para_reconocer} esta el secreto del universo.